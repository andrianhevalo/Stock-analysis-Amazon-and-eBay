\documentclass[]{article}
\usepackage{lmodern}
\usepackage{amssymb,amsmath}
\usepackage{ifxetex,ifluatex}
\usepackage{fixltx2e} % provides \textsubscript
\ifnum 0\ifxetex 1\fi\ifluatex 1\fi=0 % if pdftex
  \usepackage[T1]{fontenc}
  \usepackage[utf8]{inputenc}
\else % if luatex or xelatex
  \ifxetex
    \usepackage{mathspec}
  \else
    \usepackage{fontspec}
  \fi
  \defaultfontfeatures{Ligatures=TeX,Scale=MatchLowercase}
\fi
% use upquote if available, for straight quotes in verbatim environments
\IfFileExists{upquote.sty}{\usepackage{upquote}}{}
% use microtype if available
\IfFileExists{microtype.sty}{%
\usepackage{microtype}
\UseMicrotypeSet[protrusion]{basicmath} % disable protrusion for tt fonts
}{}
\usepackage[margin=1in]{geometry}
\usepackage{hyperref}
\hypersetup{unicode=true,
            pdftitle={Amazon and Ebay stocks analysis},
            pdfborder={0 0 0},
            breaklinks=true}
\urlstyle{same}  % don't use monospace font for urls
\usepackage{color}
\usepackage{fancyvrb}
\newcommand{\VerbBar}{|}
\newcommand{\VERB}{\Verb[commandchars=\\\{\}]}
\DefineVerbatimEnvironment{Highlighting}{Verbatim}{commandchars=\\\{\}}
% Add ',fontsize=\small' for more characters per line
\usepackage{framed}
\definecolor{shadecolor}{RGB}{248,248,248}
\newenvironment{Shaded}{\begin{snugshade}}{\end{snugshade}}
\newcommand{\KeywordTok}[1]{\textcolor[rgb]{0.13,0.29,0.53}{\textbf{#1}}}
\newcommand{\DataTypeTok}[1]{\textcolor[rgb]{0.13,0.29,0.53}{#1}}
\newcommand{\DecValTok}[1]{\textcolor[rgb]{0.00,0.00,0.81}{#1}}
\newcommand{\BaseNTok}[1]{\textcolor[rgb]{0.00,0.00,0.81}{#1}}
\newcommand{\FloatTok}[1]{\textcolor[rgb]{0.00,0.00,0.81}{#1}}
\newcommand{\ConstantTok}[1]{\textcolor[rgb]{0.00,0.00,0.00}{#1}}
\newcommand{\CharTok}[1]{\textcolor[rgb]{0.31,0.60,0.02}{#1}}
\newcommand{\SpecialCharTok}[1]{\textcolor[rgb]{0.00,0.00,0.00}{#1}}
\newcommand{\StringTok}[1]{\textcolor[rgb]{0.31,0.60,0.02}{#1}}
\newcommand{\VerbatimStringTok}[1]{\textcolor[rgb]{0.31,0.60,0.02}{#1}}
\newcommand{\SpecialStringTok}[1]{\textcolor[rgb]{0.31,0.60,0.02}{#1}}
\newcommand{\ImportTok}[1]{#1}
\newcommand{\CommentTok}[1]{\textcolor[rgb]{0.56,0.35,0.01}{\textit{#1}}}
\newcommand{\DocumentationTok}[1]{\textcolor[rgb]{0.56,0.35,0.01}{\textbf{\textit{#1}}}}
\newcommand{\AnnotationTok}[1]{\textcolor[rgb]{0.56,0.35,0.01}{\textbf{\textit{#1}}}}
\newcommand{\CommentVarTok}[1]{\textcolor[rgb]{0.56,0.35,0.01}{\textbf{\textit{#1}}}}
\newcommand{\OtherTok}[1]{\textcolor[rgb]{0.56,0.35,0.01}{#1}}
\newcommand{\FunctionTok}[1]{\textcolor[rgb]{0.00,0.00,0.00}{#1}}
\newcommand{\VariableTok}[1]{\textcolor[rgb]{0.00,0.00,0.00}{#1}}
\newcommand{\ControlFlowTok}[1]{\textcolor[rgb]{0.13,0.29,0.53}{\textbf{#1}}}
\newcommand{\OperatorTok}[1]{\textcolor[rgb]{0.81,0.36,0.00}{\textbf{#1}}}
\newcommand{\BuiltInTok}[1]{#1}
\newcommand{\ExtensionTok}[1]{#1}
\newcommand{\PreprocessorTok}[1]{\textcolor[rgb]{0.56,0.35,0.01}{\textit{#1}}}
\newcommand{\AttributeTok}[1]{\textcolor[rgb]{0.77,0.63,0.00}{#1}}
\newcommand{\RegionMarkerTok}[1]{#1}
\newcommand{\InformationTok}[1]{\textcolor[rgb]{0.56,0.35,0.01}{\textbf{\textit{#1}}}}
\newcommand{\WarningTok}[1]{\textcolor[rgb]{0.56,0.35,0.01}{\textbf{\textit{#1}}}}
\newcommand{\AlertTok}[1]{\textcolor[rgb]{0.94,0.16,0.16}{#1}}
\newcommand{\ErrorTok}[1]{\textcolor[rgb]{0.64,0.00,0.00}{\textbf{#1}}}
\newcommand{\NormalTok}[1]{#1}
\usepackage{graphicx,grffile}
\makeatletter
\def\maxwidth{\ifdim\Gin@nat@width>\linewidth\linewidth\else\Gin@nat@width\fi}
\def\maxheight{\ifdim\Gin@nat@height>\textheight\textheight\else\Gin@nat@height\fi}
\makeatother
% Scale images if necessary, so that they will not overflow the page
% margins by default, and it is still possible to overwrite the defaults
% using explicit options in \includegraphics[width, height, ...]{}
\setkeys{Gin}{width=\maxwidth,height=\maxheight,keepaspectratio}
\IfFileExists{parskip.sty}{%
\usepackage{parskip}
}{% else
\setlength{\parindent}{0pt}
\setlength{\parskip}{6pt plus 2pt minus 1pt}
}
\setlength{\emergencystretch}{3em}  % prevent overfull lines
\providecommand{\tightlist}{%
  \setlength{\itemsep}{0pt}\setlength{\parskip}{0pt}}
\setcounter{secnumdepth}{0}
% Redefines (sub)paragraphs to behave more like sections
\ifx\paragraph\undefined\else
\let\oldparagraph\paragraph
\renewcommand{\paragraph}[1]{\oldparagraph{#1}\mbox{}}
\fi
\ifx\subparagraph\undefined\else
\let\oldsubparagraph\subparagraph
\renewcommand{\subparagraph}[1]{\oldsubparagraph{#1}\mbox{}}
\fi

%%% Use protect on footnotes to avoid problems with footnotes in titles
\let\rmarkdownfootnote\footnote%
\def\footnote{\protect\rmarkdownfootnote}

%%% Change title format to be more compact
\usepackage{titling}

% Create subtitle command for use in maketitle
\newcommand{\subtitle}[1]{
  \posttitle{
    \begin{center}\large#1\end{center}
    }
}

\setlength{\droptitle}{-2em}

  \title{Amazon and Ebay stocks analysis}
    \pretitle{\vspace{\droptitle}\centering\huge}
  \posttitle{\par}
    \author{}
    \preauthor{}\postauthor{}
    \date{}
    \predate{}\postdate{}
  

\begin{document}
\maketitle

A technical analysis of two big stocks - \(\textbf{Amazon and Ebay.}\)\n

\(\textbf{This is final project on Financial Analytics course.}\)\n

The aim of this project is to apply in practise tools for analyzing
financial data. In this project, we used Random Walk theory for
simulating prices, calculated stock returns, vizualized stock returns
and calulated the correlation, found and vizualized moving avarage.

\(\textbf{Part 1.}\)

Installing all packages if needed

\begin{Shaded}
\begin{Highlighting}[]
\ControlFlowTok{if}\NormalTok{ (}\OperatorTok{!}\KeywordTok{require}\NormalTok{(}\StringTok{"quantmod"}\NormalTok{) }\OperatorTok{&}\StringTok{ }\OperatorTok{!}\KeywordTok{require}\NormalTok{(}\StringTok{"xts"}\NormalTok{) }\OperatorTok{&}\StringTok{ }\OperatorTok{!}\KeywordTok{require}\NormalTok{(}\StringTok{"rvest"}\NormalTok{) }\OperatorTok{&}\StringTok{ }\OperatorTok{!}\KeywordTok{require}\NormalTok{(}\StringTok{"stringr"}\NormalTok{) }\OperatorTok{&}\StringTok{ }\OperatorTok{!}\KeywordTok{require}\NormalTok{(}\StringTok{"forcats"}\NormalTok{) }\OperatorTok{&}\StringTok{ }\OperatorTok{!}\KeywordTok{require}\NormalTok{(}\StringTok{"lubridate"}\NormalTok{) }\OperatorTok{&}\StringTok{ }\OperatorTok{!}\KeywordTok{require}\NormalTok{(}\StringTok{"plotly"}\NormalTok{) }\OperatorTok{&}\StringTok{ }\OperatorTok{!}\KeywordTok{require}\NormalTok{(}\StringTok{"corrplot"}\NormalTok{) }\OperatorTok{&}\StringTok{ }\OperatorTok{!}\KeywordTok{require}\NormalTok{(}\StringTok{"dplyr"}\NormalTok{) }\OperatorTok{&}\StringTok{ }\OperatorTok{!}\KeywordTok{require}\NormalTok{(}\StringTok{"PerformanceAnalytics"}\NormalTok{)) \{}
    \KeywordTok{library}\NormalTok{(quantmod)}
    \KeywordTok{library}\NormalTok{(xts)}
    \KeywordTok{library}\NormalTok{(rvest)}
    \KeywordTok{library}\NormalTok{(stringr)}
    \KeywordTok{library}\NormalTok{(plotly)}
    \KeywordTok{library}\NormalTok{(corrplot)}
    \KeywordTok{library}\NormalTok{(dplyr)}
    \KeywordTok{library}\NormalTok{(PerformanceAnalytics)}
    \KeywordTok{library}\NormalTok{(magrittr)}
    \KeywordTok{library}\NormalTok{(webshot)}
\NormalTok{\}}
\end{Highlighting}
\end{Shaded}

\begin{verbatim}
## Loading required package: quantmod
\end{verbatim}

\begin{verbatim}
## Warning: package 'quantmod' was built under R version 3.5.2
\end{verbatim}

\begin{verbatim}
## Loading required package: xts
\end{verbatim}

\begin{verbatim}
## Loading required package: zoo
\end{verbatim}

\begin{verbatim}
## 
## Attaching package: 'zoo'
\end{verbatim}

\begin{verbatim}
## The following objects are masked from 'package:base':
## 
##     as.Date, as.Date.numeric
\end{verbatim}

\begin{verbatim}
## Loading required package: TTR
\end{verbatim}

\begin{verbatim}
## Version 0.4-0 included new data defaults. See ?getSymbols.
\end{verbatim}

\begin{verbatim}
## Loading required package: rvest
\end{verbatim}

\begin{verbatim}
## Warning: package 'rvest' was built under R version 3.5.2
\end{verbatim}

\begin{verbatim}
## Loading required package: xml2
\end{verbatim}

\begin{verbatim}
## Loading required package: stringr
\end{verbatim}

\begin{verbatim}
## Warning: package 'stringr' was built under R version 3.5.2
\end{verbatim}

\begin{verbatim}
## Loading required package: forcats
\end{verbatim}

\begin{verbatim}
## Warning: package 'forcats' was built under R version 3.5.2
\end{verbatim}

\begin{verbatim}
## Loading required package: lubridate
\end{verbatim}

\begin{verbatim}
## 
## Attaching package: 'lubridate'
\end{verbatim}

\begin{verbatim}
## The following object is masked from 'package:base':
## 
##     date
\end{verbatim}

\begin{verbatim}
## Loading required package: plotly
\end{verbatim}

\begin{verbatim}
## Loading required package: ggplot2
\end{verbatim}

\begin{verbatim}
## 
## Attaching package: 'plotly'
\end{verbatim}

\begin{verbatim}
## The following object is masked from 'package:ggplot2':
## 
##     last_plot
\end{verbatim}

\begin{verbatim}
## The following object is masked from 'package:stats':
## 
##     filter
\end{verbatim}

\begin{verbatim}
## The following object is masked from 'package:graphics':
## 
##     layout
\end{verbatim}

\begin{verbatim}
## Loading required package: corrplot
\end{verbatim}

\begin{verbatim}
## corrplot 0.84 loaded
\end{verbatim}

\begin{verbatim}
## Loading required package: dplyr
\end{verbatim}

\begin{verbatim}
## Warning: package 'dplyr' was built under R version 3.5.2
\end{verbatim}

\begin{verbatim}
## 
## Attaching package: 'dplyr'
\end{verbatim}

\begin{verbatim}
## The following objects are masked from 'package:lubridate':
## 
##     intersect, setdiff, union
\end{verbatim}

\begin{verbatim}
## The following objects are masked from 'package:xts':
## 
##     first, last
\end{verbatim}

\begin{verbatim}
## The following objects are masked from 'package:stats':
## 
##     filter, lag
\end{verbatim}

\begin{verbatim}
## The following objects are masked from 'package:base':
## 
##     intersect, setdiff, setequal, union
\end{verbatim}

\begin{verbatim}
## Loading required package: PerformanceAnalytics
\end{verbatim}

\begin{verbatim}
## 
## Attaching package: 'PerformanceAnalytics'
\end{verbatim}

\begin{verbatim}
## The following object is masked from 'package:graphics':
## 
##     legend
\end{verbatim}

\begin{Shaded}
\begin{Highlighting}[]
\NormalTok{webshot}\OperatorTok{::}\KeywordTok{install_phantomjs}\NormalTok{()}
\end{Highlighting}
\end{Shaded}

\begin{verbatim}
## phantomjs has been installed to /Users/andrianhevalo/Library/Application Support/PhantomJS
\end{verbatim}

\(\textbf{Part 2.}\)

\(\textbf{Stock overal performance}\)

Amazon's ticket symbol is AMZN and Ebay's is EBAY

\(\textbf{All stock data we loaded from Yahoo Finance!}\)

source: \url{https://finance.yahoo.com}

We applied getSymbols function from quantmod package having three
arguments (ticket symbol, time taken, source).

In this section we are getting two chart for both AMZN and EBAY. First
one shows stock's prices, second one is more complicated. It shows the
Bollinger Band chart, \% Bollinger change, Volume Traded and Moving
Average Convergence Diverence in 2018 alone. Those charts are widely
used in technical analysis of stocks, especially in measuring stock's
volatiliy.

The Volume chart shows how its stocks are traded on the daily.

The chartsare usually used to decide whether to buy/sell a stock.

\(\textbf{If it falls below the line, it is time to sell. If it rises above the line, it is experiencing an upward momentum.}\)

Brief explanation of what head(AMZN/EBAY) gives us:\n

Open is the price of the stock at the beginning of the trading day, high
is the highest price of the stock on that trading day, low the lowest
price of the stock on that trading day, and close the price of the stock
at closing time. Volume indicates how many stocks were traded. Adjusted
close is the closing price of the stock that adjusts the price of the
stock for corporate actions.

We coclude that Ebay stock does not worth buying in 2018. Amazon stock
is was worth buying from May till Oct 2018 (for more details look at
charts). What happens in 2019? We consider that the trends will stay the
same.

\begin{Shaded}
\begin{Highlighting}[]
\CommentTok{#start from 2018-01-01 til 2019}
\NormalTok{start <-}\StringTok{ }\KeywordTok{as.Date}\NormalTok{(}\StringTok{"2018-01-01"}\NormalTok{)}
\NormalTok{end <-}\StringTok{ }\KeywordTok{as.Date}\NormalTok{(}\StringTok{"2019-10-01"}\NormalTok{)}

\CommentTok{#getting stock data from yahoo}
\KeywordTok{getSymbols}\NormalTok{(}\StringTok{"EBAY"}\NormalTok{,}\DataTypeTok{from=}\NormalTok{start,}\DataTypeTok{to=}\NormalTok{end, }\DataTypeTok{src =} \StringTok{"yahoo"}\NormalTok{)}
\end{Highlighting}
\end{Shaded}

\begin{verbatim}
## 'getSymbols' currently uses auto.assign=TRUE by default, but will
## use auto.assign=FALSE in 0.5-0. You will still be able to use
## 'loadSymbols' to automatically load data. getOption("getSymbols.env")
## and getOption("getSymbols.auto.assign") will still be checked for
## alternate defaults.
## 
## This message is shown once per session and may be disabled by setting 
## options("getSymbols.warning4.0"=FALSE). See ?getSymbols for details.
\end{verbatim}

\begin{verbatim}
## [1] "EBAY"
\end{verbatim}

\begin{Shaded}
\begin{Highlighting}[]
\CommentTok{#Ebay's prices chart}
\NormalTok{EBAY}\OperatorTok\KeywordTok{Ad}\NormalTok{()}\OperatorTok\KeywordTok{chartSeries}\NormalTok{()}
\end{Highlighting}
\end{Shaded}

\includegraphics{financial_analysis_files/figure-latex/unnamed-chunk-2-1.pdf}

\begin{Shaded}
\begin{Highlighting}[]
\CommentTok{#Bollinger Band chart for Ebay}
\NormalTok{EBAY}\OperatorTok\KeywordTok{chartSeries}\NormalTok{(}\DataTypeTok{TA=}\StringTok{'addBBands();addVo();addMACD()'}\NormalTok{,}\DataTypeTok{subset=}\StringTok{'2018'}\NormalTok{)}
\end{Highlighting}
\end{Shaded}

\includegraphics{financial_analysis_files/figure-latex/unnamed-chunk-2-2.pdf}

\begin{Shaded}
\begin{Highlighting}[]
\CommentTok{#getting stock data from yahoo}
\KeywordTok{getSymbols}\NormalTok{(}\StringTok{"AMZN"}\NormalTok{,}\DataTypeTok{from=}\NormalTok{start,}\DataTypeTok{to=}\NormalTok{end, }\DataTypeTok{src =} \StringTok{"yahoo"}\NormalTok{)}
\end{Highlighting}
\end{Shaded}

\begin{verbatim}
## [1] "AMZN"
\end{verbatim}

\begin{Shaded}
\begin{Highlighting}[]
\CommentTok{#Amazon's prices chart}
\NormalTok{AMZN}\OperatorTok\KeywordTok{Ad}\NormalTok{()}\OperatorTok\KeywordTok{chartSeries}\NormalTok{()}
\end{Highlighting}
\end{Shaded}

\includegraphics{financial_analysis_files/figure-latex/unnamed-chunk-2-3.pdf}

\begin{Shaded}
\begin{Highlighting}[]
\CommentTok{#Bollinger Band chart for Amazon}
\NormalTok{AMZN}\OperatorTok\KeywordTok{chartSeries}\NormalTok{(}\DataTypeTok{TA=}\StringTok{'addBBands();addVo();addMACD()'}\NormalTok{,}\DataTypeTok{subset=}\StringTok{'2018'}\NormalTok{)}
\end{Highlighting}
\end{Shaded}

\includegraphics{financial_analysis_files/figure-latex/unnamed-chunk-2-4.pdf}

\begin{Shaded}
\begin{Highlighting}[]
\CommentTok{#get first few rows}
\KeywordTok{head}\NormalTok{(AMZN)}
\end{Highlighting}
\end{Shaded}

\begin{verbatim}
##            AMZN.Open AMZN.High AMZN.Low AMZN.Close AMZN.Volume
## 2018-01-02   1172.00   1190.00  1170.51    1189.01     2694500
## 2018-01-03   1188.30   1205.49  1188.30    1204.20     3108800
## 2018-01-04   1205.00   1215.87  1204.66    1209.59     3022100
## 2018-01-05   1217.51   1229.14  1210.00    1229.14     3544700
## 2018-01-08   1236.00   1253.08  1232.03    1246.87     4279500
## 2018-01-09   1256.90   1259.33  1241.76    1252.70     3661300
##            AMZN.Adjusted
## 2018-01-02       1189.01
## 2018-01-03       1204.20
## 2018-01-04       1209.59
## 2018-01-05       1229.14
## 2018-01-08       1246.87
## 2018-01-09       1252.70
\end{verbatim}

\begin{Shaded}
\begin{Highlighting}[]
\KeywordTok{head}\NormalTok{(EBAY)}
\end{Highlighting}
\end{Shaded}

\begin{verbatim}
##            EBAY.Open EBAY.High EBAY.Low EBAY.Close EBAY.Volume
## 2018-01-02     38.17     38.36    37.92      38.06     6997300
## 2018-01-03     37.99     39.28    37.90      39.22     9134400
## 2018-01-04     39.42     39.77    38.47      38.57     8958600
## 2018-01-05     38.85     39.84    38.81      39.69     7290400
## 2018-01-08     39.55     40.08    39.44      39.80     9714200
## 2018-01-09     40.10     40.13    39.48      39.53     6215900
##            EBAY.Adjusted
## 2018-01-02      37.91821
## 2018-01-03      39.07389
## 2018-01-04      38.42631
## 2018-01-05      39.54214
## 2018-01-08      39.65173
## 2018-01-09      39.38273
\end{verbatim}

Another way of vizualization

Financial data is mostly plotted with a Japanese candlestick plot, so
named because it was first created by 18th century Japanese rice
traders. We applied the function candleChart() from quantmod to create
such a chart. Also, Candlestick chats are widely used in technical
analysis to make trading decisions.

for more details go here: \url{https://plot.ly/r/candlestick-charts/}

This r code gives us two charts all having two candlesticks - blue and
red. First one indicates a day where the closing price was higher than
the open, another - day where the open was higher than the close.

\begin{Shaded}
\begin{Highlighting}[]
\KeywordTok{candleChart}\NormalTok{(AMZN, }\DataTypeTok{up.col =} \StringTok{"blue"}\NormalTok{, }\DataTypeTok{dn.col =} \StringTok{"red"}\NormalTok{, }\DataTypeTok{theme =} \StringTok{"white"}\NormalTok{)}
\end{Highlighting}
\end{Shaded}

\includegraphics{financial_analysis_files/figure-latex/unnamed-chunk-3-1.pdf}

\begin{Shaded}
\begin{Highlighting}[]
\KeywordTok{candleChart}\NormalTok{(EBAY, }\DataTypeTok{up.col =} \StringTok{"blue"}\NormalTok{, }\DataTypeTok{dn.col =} \StringTok{"red"}\NormalTok{, }\DataTypeTok{theme =} \StringTok{"white"}\NormalTok{)}
\end{Highlighting}
\end{Shaded}

\includegraphics{financial_analysis_files/figure-latex/unnamed-chunk-3-2.pdf}

\(\textbf{Below, we plot AMZN's and EBAY's adjusted close together.}\)

We initialize xts object from quantmod containing stock prices for AMZN
and EBAY.

Then, we create a plot using zoo method which allows for multiple series
be ploted on the same plot.

\begin{Shaded}
\begin{Highlighting}[]
\NormalTok{stocks <-}\StringTok{ }\KeywordTok{as.xts}\NormalTok{(}\KeywordTok{data.frame}\NormalTok{(}\DataTypeTok{AMZN =}\NormalTok{ AMZN[, }\StringTok{"AMZN.Close"}\NormalTok{], }\DataTypeTok{EBAY =}\NormalTok{ EBAY[, }\StringTok{"EBAY.Close"}\NormalTok{]))}
\KeywordTok{head}\NormalTok{(stocks)}
\end{Highlighting}
\end{Shaded}

\begin{verbatim}
##            AMZN.Close EBAY.Close
## 2018-01-02    1189.01      38.06
## 2018-01-03    1204.20      39.22
## 2018-01-04    1209.59      38.57
## 2018-01-05    1229.14      39.69
## 2018-01-08    1246.87      39.80
## 2018-01-09    1252.70      39.53
\end{verbatim}

\begin{Shaded}
\begin{Highlighting}[]
\KeywordTok{plot}\NormalTok{(}\KeywordTok{as.zoo}\NormalTok{(stocks[, }\KeywordTok{c}\NormalTok{(}\StringTok{"AMZN.Close"}\NormalTok{)]))}
\KeywordTok{par}\NormalTok{(}\DataTypeTok{new =} \OtherTok{TRUE}\NormalTok{)}

\KeywordTok{plot}\NormalTok{(}\KeywordTok{as.zoo}\NormalTok{(stocks[, }\StringTok{"EBAY.Close"}\NormalTok{]), }\DataTypeTok{screens =} \DecValTok{1}\NormalTok{, }\DataTypeTok{lty =} \DecValTok{3}\NormalTok{, }\DataTypeTok{xaxt =} \StringTok{"n"}\NormalTok{, }\DataTypeTok{yaxt =} \StringTok{"n"}\NormalTok{, }
    \DataTypeTok{xlab =} \StringTok{""}\NormalTok{, }\DataTypeTok{ylab =} \StringTok{""}\NormalTok{)}

\KeywordTok{mtext}\NormalTok{(}\StringTok{"Price"}\NormalTok{, }\DataTypeTok{side =} \DecValTok{4}\NormalTok{, }\DataTypeTok{line =} \DecValTok{3}\NormalTok{)}

\CommentTok{#adding legend to plot}
\KeywordTok{legend}\NormalTok{(}\StringTok{"topleft"}\NormalTok{, }\KeywordTok{c}\NormalTok{(}\StringTok{"AMZN"}\NormalTok{, }\StringTok{"EBAY"}\NormalTok{), }\DataTypeTok{lty =} \DecValTok{1}\OperatorTok{:}\DecValTok{3}\NormalTok{, }\DataTypeTok{cex =} \FloatTok{0.5}\NormalTok{)}
\end{Highlighting}
\end{Shaded}

\includegraphics{financial_analysis_files/figure-latex/unnamed-chunk-4-1.pdf}

\(\textbf{Part 3.}\)

\(\textbf{Visualization of a Correlation Matrix}\)

To successfully purchase a stock, we should take into account
correlation (the smaller correlation is, the more bigger is rate of
return).

\(\textbf{A rule of thumb: do not put all your eggs in one basket!}\)

Since, AMZN and EBAY are stocks from different sectrors, therefore the
correlation is small.

Investing money in stocks from different sectros minimizes the risk.

\begin{Shaded}
\begin{Highlighting}[]
\CommentTok{#merging two data frames together}
\NormalTok{data<-}\KeywordTok{cbind}\NormalTok{(}\KeywordTok{diff}\NormalTok{(}\KeywordTok{log}\NormalTok{(}\KeywordTok{Cl}\NormalTok{(EBAY))),}\KeywordTok{diff}\NormalTok{(}\KeywordTok{log}\NormalTok{(}\KeywordTok{Cl}\NormalTok{(AMZN))))}

\CommentTok{#getting chart with correlation data}
\KeywordTok{chart.Correlation}\NormalTok{(data)}
\end{Highlighting}
\end{Shaded}

\includegraphics{financial_analysis_files/figure-latex/unnamed-chunk-5-1.pdf}

\(\textbf{Amazon and Ebay stocks returns}\)

To do that, we need to use apply function which transforms our data into
appropriate type. We calculate the returns in certain period:
\(return_{t, 0} = \frac{price_t}{price_0}\)

The plot shows us the profitability of each stocks. It seen clearly seen
in the plot that two stocks are not correlated what we've already shown
in previuos section. Amazon stock is much more profitable than Ebay.

In addition, we could also calculate and vizualize stock changes.

\(change_t = log(price_t) - log(price_{t - 1})\) t - period of time.

\begin{Shaded}
\begin{Highlighting}[]
\CommentTok{#getting stock returns}
\NormalTok{stock_returns =}\StringTok{ }\KeywordTok{apply}\NormalTok{(stocks, }\DecValTok{1}\NormalTok{, }\ControlFlowTok{function}\NormalTok{(x) \{x }\OperatorTok{/}\StringTok{ }\NormalTok{stocks[}\DecValTok{1}\NormalTok{,]\}) }\OperatorTok\StringTok{ }
\StringTok{                                            }\NormalTok{t }\OperatorTok\StringTok{ }\NormalTok{as.xts}

\CommentTok{#getting stock change using diff function}
\NormalTok{stock_change =}\StringTok{ }\KeywordTok{diff}\NormalTok{(}\KeywordTok{log}\NormalTok{(stock_returns[, ]))}

\CommentTok{#heading data}
\KeywordTok{head}\NormalTok{(stock_change)}
\end{Highlighting}
\end{Shaded}

\begin{verbatim}
##             AMZN.Close   EBAY.Close
## 2018-01-02          NA           NA
## 2018-01-03 0.012694369  0.030022958
## 2018-01-04 0.004466026 -0.016712074
## 2018-01-05 0.016033319  0.028624469
## 2018-01-08 0.014321657  0.002767646
## 2018-01-09 0.004664776 -0.006807035
\end{verbatim}

\begin{Shaded}
\begin{Highlighting}[]
\KeywordTok{head}\NormalTok{(stock_returns)}
\end{Highlighting}
\end{Shaded}

\begin{verbatim}
##            AMZN.Close EBAY.Close
## 2018-01-02   1.000000   1.000000
## 2018-01-03   1.012775   1.030478
## 2018-01-04   1.017308   1.013400
## 2018-01-05   1.033751   1.042827
## 2018-01-08   1.048662   1.045717
## 2018-01-09   1.053566   1.038623
\end{verbatim}

\begin{Shaded}
\begin{Highlighting}[]
\KeywordTok{plot}\NormalTok{(}\KeywordTok{as.zoo}\NormalTok{(stock_returns), }\DataTypeTok{screens =} \DecValTok{1}\NormalTok{, }\DataTypeTok{lty =} \DecValTok{1}\OperatorTok{:}\DecValTok{3}\NormalTok{, }\DataTypeTok{xlab =} \StringTok{"Date"}\NormalTok{, }\DataTypeTok{ylab =} \StringTok{"Return"}\NormalTok{, }\DataTypeTok{main =} \StringTok{"Amazon and Ebay stock returns"}\NormalTok{)}

\KeywordTok{legend}\NormalTok{(}\StringTok{"topleft"}\NormalTok{, }\KeywordTok{c}\NormalTok{(}\StringTok{"AMZN"}\NormalTok{, }\StringTok{"EBAY"}\NormalTok{), }\DataTypeTok{lty =} \DecValTok{1}\OperatorTok{:}\DecValTok{3}\NormalTok{, }\DataTypeTok{cex =} \FloatTok{0.5}\NormalTok{)}
\end{Highlighting}
\end{Shaded}

\includegraphics{financial_analysis_files/figure-latex/unnamed-chunk-6-1.pdf}

\begin{Shaded}
\begin{Highlighting}[]
\KeywordTok{plot}\NormalTok{(}\KeywordTok{as.zoo}\NormalTok{(stock_change), }\DataTypeTok{screens =} \DecValTok{1}\NormalTok{, }\DataTypeTok{lty =} \DecValTok{1}\OperatorTok{:}\DecValTok{3}\NormalTok{, }\DataTypeTok{xlab =} \StringTok{"Date"}\NormalTok{, }\DataTypeTok{ylab =} \StringTok{"Log Difference"}\NormalTok{, }\DataTypeTok{main =} \StringTok{"Amazon and Ebay stock change"}\NormalTok{)}

\KeywordTok{legend}\NormalTok{(}\StringTok{"topleft"}\NormalTok{, }\KeywordTok{c}\NormalTok{(}\StringTok{"AMZN"}\NormalTok{, }\StringTok{"Ebay"}\NormalTok{), }\DataTypeTok{lty =} \DecValTok{1}\OperatorTok{:}\DecValTok{3}\NormalTok{, }\DataTypeTok{cex =} \FloatTok{0.5}\NormalTok{)}
\end{Highlighting}
\end{Shaded}

\includegraphics{financial_analysis_files/figure-latex/unnamed-chunk-6-2.pdf}

\(\textbf{Moving Avarages}\)

Moving avarages smooth a series and help find trends in stocks. They do
not predict price direction, but rather define the current direction.
Most moving averages are based on closing prices.

There two types of Moving avarages - Simple and Exponential. Simple one
calculates the avarage price over certain periods of time. Exponential -
reduce the lag by applying more weight to recent prices.

\(MA^q_t = \frac{1}{q}\sum_{i=0}{q - 1}x_{t-i}\)

\begin{Shaded}
\begin{Highlighting}[]
\NormalTok{start =}\StringTok{ }\KeywordTok{as.Date}\NormalTok{(}\StringTok{"2018-01-01"}\NormalTok{)}
\KeywordTok{getSymbols}\NormalTok{(}\KeywordTok{c}\NormalTok{(}\StringTok{"AMZN"}\NormalTok{, }\StringTok{"EBAY"}\NormalTok{), }\DataTypeTok{src =} \StringTok{"yahoo"}\NormalTok{, }\DataTypeTok{from =}\NormalTok{ start, }\DataTypeTok{to =}\NormalTok{ end)}
\end{Highlighting}
\end{Shaded}

\begin{verbatim}
## [1] "AMZN" "EBAY"
\end{verbatim}

\begin{Shaded}
\begin{Highlighting}[]
\KeywordTok{candleChart}\NormalTok{(AMZN, }\DataTypeTok{up.col =} \StringTok{"black"}\NormalTok{, }\DataTypeTok{dn.col =} \StringTok{"red"}\NormalTok{, }\DataTypeTok{theme =} \StringTok{"white"}\NormalTok{, }\DataTypeTok{subset =} \StringTok{"2018-01-01/"}\NormalTok{, ad)}
\end{Highlighting}
\end{Shaded}

\includegraphics{financial_analysis_files/figure-latex/unnamed-chunk-7-1.pdf}

\begin{Shaded}
\begin{Highlighting}[]
\CommentTok{#adding MA via addSMA from quantmod}
\KeywordTok{addSMA}\NormalTok{(}\DataTypeTok{n =} \DecValTok{20}\NormalTok{)}
\end{Highlighting}
\end{Shaded}

\includegraphics{financial_analysis_files/figure-latex/unnamed-chunk-7-2.pdf}

\begin{Shaded}
\begin{Highlighting}[]
\KeywordTok{addEMA}\NormalTok{(}\DataTypeTok{n =} \DecValTok{20}\NormalTok{)}
\end{Highlighting}
\end{Shaded}

\includegraphics{financial_analysis_files/figure-latex/unnamed-chunk-7-3.pdf}

\begin{Shaded}
\begin{Highlighting}[]
\KeywordTok{candleChart}\NormalTok{(EBAY, }\DataTypeTok{up.col =} \StringTok{"black"}\NormalTok{, }\DataTypeTok{dn.col =} \StringTok{"red"}\NormalTok{, }\DataTypeTok{theme =} \StringTok{"white"}\NormalTok{, }\DataTypeTok{subset =} \StringTok{"2018-01-01/"}\NormalTok{)}
\end{Highlighting}
\end{Shaded}

\includegraphics{financial_analysis_files/figure-latex/unnamed-chunk-7-4.pdf}

\begin{Shaded}
\begin{Highlighting}[]
\KeywordTok{addSMA}\NormalTok{(}\DataTypeTok{n =} \DecValTok{20}\NormalTok{)}
\end{Highlighting}
\end{Shaded}

\includegraphics{financial_analysis_files/figure-latex/unnamed-chunk-7-5.pdf}

\begin{Shaded}
\begin{Highlighting}[]
\KeywordTok{addEMA}\NormalTok{(}\DataTypeTok{n =} \DecValTok{20}\NormalTok{)}
\end{Highlighting}
\end{Shaded}

\includegraphics{financial_analysis_files/figure-latex/unnamed-chunk-7-6.pdf}

\(\textbf{Part 6}\)

Logreturns and vizualization of risk vs reward

We applied plotly to create a visualization of each stock's risk vs
reward.

Risk: standard deviation of log returns.

Reward: mean of log returns.

\begin{Shaded}
\begin{Highlighting}[]
\CommentTok{#Loads the company stock using ticker}
\KeywordTok{getSymbols}\NormalTok{(}\StringTok{"AMZN"}\NormalTok{,}\DataTypeTok{from=}\StringTok{"2008-08-01"}\NormalTok{,}\DataTypeTok{to=}\StringTok{"2018-08-20"}\NormalTok{)}
\end{Highlighting}
\end{Shaded}

\begin{verbatim}
## [1] "AMZN"
\end{verbatim}

\begin{Shaded}
\begin{Highlighting}[]
\KeywordTok{getSymbols}\NormalTok{(}\StringTok{"EBAY"}\NormalTok{,}\DataTypeTok{from=}\StringTok{"2008-08-01"}\NormalTok{,}\DataTypeTok{to=}\StringTok{"2018-08-20"}\NormalTok{)}
\end{Highlighting}
\end{Shaded}

\begin{verbatim}
## [1] "EBAY"
\end{verbatim}

\begin{Shaded}
\begin{Highlighting}[]
\CommentTok{#Stock returns in log}
\NormalTok{AMZN_log_returns<-AMZN}\OperatorTok\KeywordTok{Ad}\NormalTok{()}\OperatorTok\KeywordTok{dailyReturn}\NormalTok{(}\DataTypeTok{type=}\StringTok{'log'}\NormalTok{)}
\NormalTok{EBAY_log_returns<-EBAY}\OperatorTok\KeywordTok{Ad}\NormalTok{()}\OperatorTok\KeywordTok{dailyReturn}\NormalTok{(}\DataTypeTok{type=}\StringTok{'log'}\NormalTok{)}

\CommentTok{#Mean of log stock returns }
\NormalTok{AMZN_mean_log<-}\KeywordTok{mean}\NormalTok{(AMZN_log_returns)}
\NormalTok{EBAY_mean_log<-}\KeywordTok{mean}\NormalTok{(EBAY_log_returns)}

\CommentTok{#standard deviation of log stock returns}
\NormalTok{AMZN_sd_log<-}\KeywordTok{sd}\NormalTok{(AMZN_log_returns)}
\NormalTok{EBAY_sd_log<-}\KeywordTok{sd}\NormalTok{(EBAY_log_returns)}

\NormalTok{mean_log<-}\KeywordTok{c}\NormalTok{(AMZN_mean_log,EBAY_mean_log)}
\NormalTok{sd_log<-}\KeywordTok{c}\NormalTok{(AMZN_sd_log,EBAY_sd_log)}

\CommentTok{#create data frame}
\NormalTok{graph1<-}\KeywordTok{data.frame}\NormalTok{(}\KeywordTok{rbind}\NormalTok{(}\KeywordTok{c}\NormalTok{(}\StringTok{"AMZN"}\NormalTok{,AMZN_mean_log, AMZN_sd_log), }\KeywordTok{c}\NormalTok{(}\StringTok{"EBAY"}\NormalTok{,EBAY_mean_log,EBAY_sd_log), }\DataTypeTok{stringsAsFactors =} \OtherTok{FALSE}\NormalTok{))}

\NormalTok{graph1<-}\KeywordTok{data.frame}\NormalTok{(mean_log, sd_log)}
\KeywordTok{rownames}\NormalTok{(graph1)<-}\KeywordTok{c}\NormalTok{(}\StringTok{"AMZN"}\NormalTok{,}\StringTok{"EBAY"}\NormalTok{)}
\KeywordTok{colnames}\NormalTok{(graph1)<-}\KeywordTok{c}\NormalTok{(}\StringTok{"AMZN_log_returns"}\NormalTok{, }\StringTok{"EBAY_log_returns"}\NormalTok{)}

\NormalTok{xlab<-}\KeywordTok{list}\NormalTok{(}\DataTypeTok{title=}\StringTok{"Reward"}\NormalTok{)}
\NormalTok{ylab<-}\KeywordTok{list}\NormalTok{(}\DataTypeTok{title=}\StringTok{"Risk"}\NormalTok{)}

\KeywordTok{plot_ly}\NormalTok{(}\DataTypeTok{x=}\NormalTok{graph1[,}\DecValTok{1}\NormalTok{],}\DataTypeTok{y=}\NormalTok{graph1[,}\DecValTok{2}\NormalTok{],}\DataTypeTok{text=}\KeywordTok{rownames}\NormalTok{(graph1),}\DataTypeTok{type=}\StringTok{'scatter'}\NormalTok{,}\DataTypeTok{mode=}\StringTok{"markers"}\NormalTok{,}\DataTypeTok{marker=}\KeywordTok{list}\NormalTok{(}\DataTypeTok{color=}\KeywordTok{c}\NormalTok{(}\StringTok{"black"}\NormalTok{,}\StringTok{"blue"}\NormalTok{)))}\OperatorTok\KeywordTok{layout}\NormalTok{(}\DataTypeTok{title=}\StringTok{"Risk vs Reward"}\NormalTok{,}\DataTypeTok{xaxis=}\NormalTok{xlab,}\DataTypeTok{yaxis=}\NormalTok{ylab)}
\end{Highlighting}
\end{Shaded}

\includegraphics{financial_analysis_files/figure-latex/unnamed-chunk-8-1.pdf}

\(\textbf{Part 7}\)

Random Walk for prices simulation of Amazon stock

The random walk theory suggests that changes in stock prices have the
same distribution (normal) and are independent of each other, therefore,
the past movement or trend of a stock price or market cannot be used to
predict its future movement.

\begin{Shaded}
\begin{Highlighting}[]
\NormalTok{testsim<-}\KeywordTok{rep}\NormalTok{(}\OtherTok{NA}\NormalTok{,}\DecValTok{500}\NormalTok{)}

\CommentTok{#generate random daily exponent increase rate using AMZN's mean and sd log returns}

\CommentTok{#one year 252 trading days, simulate for 2 years }
\CommentTok{# 2*252 trading days}

\NormalTok{AMZN_prices<-}\KeywordTok{rep}\NormalTok{(}\OtherTok{NA}\NormalTok{,}\DecValTok{252}\OperatorTok{*}\DecValTok{2}\NormalTok{)}

\CommentTok{#most recent price}
\NormalTok{AMZN_prices[}\DecValTok{1}\NormalTok{]<-}\KeywordTok{as.numeric}\NormalTok{(AMZN}\OperatorTok{$}\NormalTok{AMZN.Adjusted[}\KeywordTok{length}\NormalTok{(AMZN}\OperatorTok{$}\NormalTok{AMZN.Adjusted),])}

\CommentTok{#start simulating prices}

\ControlFlowTok{for}\NormalTok{(i }\ControlFlowTok{in} \DecValTok{2}\OperatorTok{:}\DecValTok{500}\NormalTok{)\{}
  \CommentTok{#generates multivariate normal random variates in the space}
\NormalTok{  AMZN_prices[i]<-AMZN_prices[i}\OperatorTok{-}\DecValTok{1}\NormalTok{]}\OperatorTok{*}\KeywordTok{exp}\NormalTok{(}\KeywordTok{rnorm}\NormalTok{(}\DecValTok{1}\NormalTok{,AMZN_mean_log,AMZN_sd_log))}
\NormalTok{\}}

\CommentTok{#creating a frame}
\NormalTok{random_data<-}\KeywordTok{cbind}\NormalTok{(AMZN_prices,}\DecValTok{1}\OperatorTok{:}\NormalTok{(}\DecValTok{252}\OperatorTok{*}\DecValTok{2}\NormalTok{))}
\KeywordTok{colnames}\NormalTok{(random_data)<-}\KeywordTok{c}\NormalTok{(}\StringTok{"Price"}\NormalTok{,}\StringTok{"Day"}\NormalTok{)}
\NormalTok{random_data<-}\KeywordTok{as.data.frame}\NormalTok{(random_data)}

\NormalTok{random_data}\OperatorTok\KeywordTok{ggplot}\NormalTok{(}\KeywordTok{aes}\NormalTok{(Day,Price))}\OperatorTok{+}\KeywordTok{geom_line}\NormalTok{()}\OperatorTok{+}\KeywordTok{labs}\NormalTok{(}\DataTypeTok{title=}\StringTok{"Amazon (AMZN) price simulation for 2 years"}\NormalTok{)}\OperatorTok{+}\KeywordTok{theme_bw}\NormalTok{()}
\end{Highlighting}
\end{Shaded}

\begin{verbatim}
## Warning: Removed 4 rows containing missing values (geom_path).
\end{verbatim}

\includegraphics{financial_analysis_files/figure-latex/unnamed-chunk-9-1.pdf}

\begin{Shaded}
\begin{Highlighting}[]
\CommentTok{#ending session}
\KeywordTok{sessionInfo}\NormalTok{()}
\end{Highlighting}
\end{Shaded}

\begin{verbatim}
## R version 3.5.1 (2018-07-02)
## Platform: x86_64-apple-darwin15.6.0 (64-bit)
## Running under: macOS  10.14.4
## 
## Matrix products: default
## BLAS: /Library/Frameworks/R.framework/Versions/3.5/Resources/lib/libRblas.0.dylib
## LAPACK: /Library/Frameworks/R.framework/Versions/3.5/Resources/lib/libRlapack.dylib
## 
## locale:
## [1] en_US.UTF-8/en_US.UTF-8/en_US.UTF-8/C/en_US.UTF-8/en_US.UTF-8
## 
## attached base packages:
## [1] stats     graphics  grDevices utils     datasets  methods   base     
## 
## other attached packages:
##  [1] PerformanceAnalytics_1.5.2 dplyr_0.8.0.1             
##  [3] corrplot_0.84              plotly_4.9.0              
##  [5] ggplot2_3.1.0              lubridate_1.7.4           
##  [7] forcats_0.4.0              stringr_1.4.0             
##  [9] rvest_0.3.3                xml2_1.2.0                
## [11] quantmod_0.4-14            TTR_0.23-4                
## [13] xts_0.11-2                 zoo_1.8-4                 
## 
## loaded via a namespace (and not attached):
##  [1] tidyselect_0.2.5  purrr_0.2.5       lattice_0.20-35  
##  [4] colorspace_1.3-2  htmltools_0.3.6   viridisLite_0.3.0
##  [7] yaml_2.2.0        rlang_0.3.4       pillar_1.3.1     
## [10] later_0.8.0       glue_1.3.0        withr_2.1.2      
## [13] plyr_1.8.4        munsell_0.5.0     gtable_0.2.0     
## [16] htmlwidgets_1.3   evaluate_0.12     labeling_0.3     
## [19] knitr_1.20        callr_3.1.1       ps_1.3.0         
## [22] httpuv_1.5.1      crosstalk_1.0.0   curl_3.2         
## [25] Rcpp_1.0.0        xtable_1.8-3      scales_1.0.0     
## [28] backports_1.1.2   promises_1.0.1    webshot_0.5.1    
## [31] jsonlite_1.6      mime_0.6          digest_0.6.18    
## [34] stringi_1.2.4     processx_3.2.1    shiny_1.3.2      
## [37] grid_3.5.1        rprojroot_1.3-2   quadprog_1.5-5   
## [40] tools_3.5.1       magrittr_1.5      lazyeval_0.2.1   
## [43] tibble_2.1.1      crayon_1.3.4      tidyr_0.8.2      
## [46] pkgconfig_2.0.2   data.table_1.12.2 assertthat_0.2.0 
## [49] rmarkdown_1.10    httr_1.4.0        R6_2.3.0         
## [52] compiler_3.5.1
\end{verbatim}

\(\textbf{Conclusion:}\) Investing in stock market is not so easy as it
seems to be. The probability of losing money is high, because many
investors do not pay attention to risk. But with such tools like Random
Walk and Moving Avarage chart, we can avoid risk and invest in stock
successfully. Only being familiar with financial analysis rools will
help us make right decisions.

References:
\url{https://stockcharts.com/school/doku.php?id=chart_school:technical_indicators:moving_averages}
\url{https://thebusinessprofessor.com/knowledge-base/random-walk-theory-stock-market-explained/}
\url{https://cran.r-project.org/web/packages/magrittr/vignettes/magrittr.html}
\url{https://towardsdatascience.com/analyzing-stocks-using-r-550be7f5f20d}
\url{https://ntguardian.wordpress.com/2017/04/03/introduction-stock-market-data-r-2/}
\url{https://ntguardian.wordpress.com/2017/03/27/introduction-stock-market-data-r-1/}


\end{document}
